%\documentclass[printmode,oneside,eng]{mgr}  % do publikacji elektronicznej
\documentclass[printmode,eng]{mgr}  % do wydruku dwustronnego
%\documentclass[printmode,eng,draft]{mgr}  % do wydruku dwustronnego w~wersji roboczej

%----------------------| Input encoding configuration |

\usepackage[utf8]{inputenc}

%-------------------------| Typeface configuration |------------------------

%% Najpierw (chyba) należy wybrać czcionkę tekstu, potem matematyki

%% wybór czcionki Computer Concrete specjalnie zaprojektowanej do użycia 
%% z~czcionką matematyczną euler (nie ma ona niestety wersji bold - proponuje się więc użycie bolda z kroju
%%   Computer Modern Sans Serif)
%\usepackage{beton}             % ZALECANA DO SKŁADU TEKSTÓW PO ANGIELSKU
%\renewcommand{\bfdefault}{sbc} % to use Computer Modern Sans Serif demibold condensed fonts as bold 
%% choć może komuś bardziej podobać się będzie ten krój normalnej szerokości 
%\renewcommand{\bfdefault}{sb} % to use Computer Modern Sans Serif demibold fonts as bold 

%% alternatywny wybór Antykwy Półtawskiego - czcionki zaprojektowanej
%% specjalnie dla języka polskiego uwzględniającej jego rytm 
%\usepackage{antpolt}
%% alternatywny wybór Antykwy Toruńskiej - bardziej "współczesnej",
%% całkowicie polskiej czcionki
\usepackage{anttor}             % ZALECANA DO SKŁADU TEKSTÓW PO POLSKU
%\usepackage[math]{anttor}      % math włącza antykwę także w~matematyce
                                % ale chyba coś psuje :( (np. brak strzałek)

%% alternatywnie wybór czcionki URW Palladio
%\usepackage{newpxtext}  % Palatino font
%\linespread{1.05}  % Palatino needs more leading space (between lines)

%% ustawienie czcionki eulerowskiej do składu wyrażeń matematycznych
%\usepackage[euler-digits,small]{eulervm}
\usepackage{eulervm}

%% ustawienie czcionki bezszeryfowej Monospace (typewriter, code) font (last) - skład poleceń, wydruków programów
\usepackage[varqu,varl]{inconsolata}

%-------------------------| Sprawy polskie |------------------------

\usepackage[T1]{fontenc} % bez tego są złe znaki / { } w czcionce tt
                         % ale musi być użyty pakiet polski bez opcji
                         % wybierającej układ - gdy zamarkowane
                         % używamy pakiet polski z~opcją wyboru układu
\usepackage{polski}
%\usepackage[OT4]{polski} % domyślnie?

%---------------------------| Packetology |-------------------------

\usepackage{geometry}           % manipulowanie geometrią łamu
\usepackage{indentfirst}        % wcięcia akapitowe w pierwszych paragrafach rozdziałów
\usepackage[dvipsnames]{xcolor} % by mieć nazwy kolorów w~rodzaju \green

%% zmiana formatowania tytulariów
%\usepackage[raggedleft]{titlesec}

\usepackage[titles]{tocloft}    % do formatowani spisów

%% w razie potrzeby zmieniamy nieco odstępy przed rozdziałami i podrozdziałami w spisie
%% treści by uniknąć pojedynczego samotnego tytułu rozdziału/podrozdziału na dole/górze strony
%% i by ładnie się zmieściła notka o latechu
\addtolength{\cftbeforechapskip}{-0.9ex}
\addtolength{\cftbeforesecskip}{0.02ex}

%%modyfikacja odstępów między pozycjami w~bibliografii
\let\oldbibliography\thebibliography
\renewcommand{\thebibliography}[1]{%
  \oldbibliography{#1}%
  \setlength{\itemsep}{0pt plus 0.3ex}%
}

%% ustawienie maksymalnej liczby i obszaru dla obiektów pływających (rysunków, tabel)
\setcounter{topnumber}{3}
\setcounter{totalnumber}{4}
\renewcommand{\topfraction}{.8}

\usepackage{graphicx}                % dołaczanie i manipulowanie grafikami
\usepackage[export]{adjustbox}       % więcej komend operujących na "pudełkach"
\usepackage[caption = false]{subfig} % obsługa rysunków z częściami

\usepackage{svg}                % dołączanie grafik w formacie svg
%\usepackage[inkscapepath=svgdir]{svg}  % w nowszych wersjach

\usepackage{tikz}               % dołączanie grafik w formacie TikZ
\usepackage{makecell}           % trochę tikzowej magii
    \tikzstyle{block} = [draw, fill=blue!20, rectangle, 
     minimum height=3em, minimum width=6em]
    \tikzstyle{sum} = [draw, fill=blue!20, circle, node distance=3.5cm]
    \tikzstyle{input} = [coordinate]
    \tikzstyle{output} = [coordinate]
    \tikzstyle{pinstyle} = [pin edge={to-,thin,black}]    
    \usetikzlibrary{arrows,automata,calc,positioning}
\usepackage{standalone}  

\usepackage{pgfplots}           % do robienia wykresów 
\pgfplotsset{compat=1.5}        %% why 1.5? pl. update

\usepackage{mathtools}          % doskonałe pakiety rozszerzające do matematyki
\usepackage{amssymb, amsfonts}  % mathtools zastępuje amsmath (naprawione błedy, dodane rozszerzenia) 
\usepackage{wasysym}            % trochę więcej różnych symboli (buźki)
%\usepackage[fleqn]{mathtools}  % równania w wersji dosuniętej w lewo
%% automatyczne numerowanie jedynie tych równań, do których są odwołania w~tekście
%\mathtoolsset{showonlyrefs=true}

%% symbole do oznaczania stopek w~miejsce liczb
\renewcommand{\thefootnote}{\fnsymbol{footnote}}
%% powtórzone z latex.ltx - w przeciwnym razie znika symbol \textbardbl przy
%% wybranej Antykwie Toruńskiej (sprawdzić co z innymi definicjami z omsenc.def,
%% sprawdzić czy też przy innych czcionkach)
\DeclareTextSymbolDefault{\textbardbl}{OMS}
%% automatyczne resetowanie wartości licznika stopek użyteczne przy użyciu symboli
%% do oznaczania stopek w~miejsce liczb (liczba dostępnych symboli wynosi tylko 9)
\usepackage{etoolbox,pdftexcmds}
\makeatletter
\patchcmd{\footnote}
  {\stepcounter\@mpfn}
%  {\stepcounter\@mpfn\check@overflow\@mpfn} %%było w~przykładzie
  {\stepcounter\@mpfn\check@overflow}
  {}{}
\newcommand{\check@overflow}{%
  \ifnum\pdf@strcmp{\@mpfn}{footnote}=\z@
    \ifnum\value{footnote}>8  %tu było 9, ale czasami nie resetowało poprawnie
      \setcounter{footnote}{1}%
    \fi
  \fi
}
\makeatother

\usepackage{hyperref}           % obsługa aktywnych odnośników i hypertekstu
\usepackage{url}                % obsługa adresów url

%% pakiet minted do wydruków programów
%% 'minted' gives much better highlight then the default listings,
%% but it reqires Python with Pygments (check the documentation of minted)
%% it also requires passing '-shell-escape' option to pdflatex during compilation
\usepackage[%
  cache=false,
  chapter,
  %    outputdir=build,  % if building in separate directory, this must be included
                         % ale nie zawsze działa poprawnie :( 
    newfloat  % required if multi-page floating listings are needed (see below)
]{minted}  % (note: from my experience must be loaded before 'csquotes')
%% by wybrać inny styl - lista dostępnych styli: pygmentize -L styles
%\usemintedstyle{igor}
\SetupFloatingEnvironment{listing}{name=Wydruk}
%% kolor tła używany w~wydrukach
\definecolor{OurListingBackground}{rgb}{0.95,0.95,0.95}
%% fix the minted@colorbg environment bug
\makeatletter
\renewenvironment{minted@colorbg}[1]
 {\def\minted@bgcol{#1}%
  \noindent
  \begin{lrbox}{\minted@bgbox}
  \begin{minipage}{\linewidth-2\fboxsep}}
 {\end{minipage}%
  \end{lrbox}%
  \setlength{\topsep}{\bigskipamount}% set the vertical space
  \trivlist\item\relax % ensure going to a new line
  \colorbox{\minted@bgcol}{\usebox{\minted@bgbox}}%
  \endtrivlist % close the trivlist
 }
\makeatother

%%pakiet do robienia notatek w trakcie pracy
\usepackage{todonotes}
\makeatletter   %spolszczenie
\renewcommand{\@todonotes@todolistname}{Do zrobienia}
\renewcommand{\@todonotes@MissingFigureText}{Rysunek}
\renewcommand{\@todonotes@MissingFigureUp}{Brakujący}
\renewcommand{\@todonotes@MissingFigureDown}{rysunek}
\makeatother

%% dodaje w pliku wynikowym klucze etykiet i referencji - wygodne przy pracy nad tekstem
%\usepackage{showkeys}

%%inne przydatne
%\usepackage{fancyhdr}
%\usepackage{fancyvrb}
%\usepackage{lipsum}  
%\usepackage{listings}

%-----------------------| End of packetology |----------------------

%---------------------------| Tytularia |---------------------------

% Zmień dane stosownie do tytułu i autora pracy tutaj
\author{Jan Bronicki}
\title{Sztuczna inteligencja w roli gracza w grze zręcznościowej}
\engtitle{Artificial intelligence as a player in arcade game}
\supervisor{Dr inż. Mariusz Uchroński,\\ Katedra Automatyki, Mechatroniki i Systemów Sterowania}
\field{Automatyka i~Robotyka (AIR)}
\specialisation{Przemysł 4.0 (ARP)}
\date{2022}

%-------------------------| Hyper data general configuration |------------------------

\hypersetup{unicode,
   pdfpagemode=UseOutlines,   % otwiera dokument w trybie jednej strony
   pdfpagelayout=SinglePage,  %
   pdfstartpage=1,            % na podanej stronie
   bookmarksopen=true,        % rozwinięcie zakładek
   bookmarksopenlevel=1,      % do jakiego poziomu
   colorlinks=true,      % kolorowanie odnośników zamiast ramki wokół nich
   citecolor=cyan,       % kolor odnośników do bibliografii, domyślnie zielony
   filecolor=red,        % kolor odnośników do lokalnych plików, domyśnie magenta
   linkcolor=blue,       % kolor odnośników wewnętrznych, domyślnie czerwony
   menucolor=green,      % kolor pozycji menu Acrobata, domyślnie czerwony
   urlcolor=blue,        % kolor odnośników do adresów internetowych, domyślnie cyan
                              % DANE DOKUMENTACJI
   pdftitle={Bąk jaki jest każdy widzi. Studium zachowań - Przykład
    i~wytyczne formatowania pracy dyplomowej},
   pdfauthor={Roberto Orozco, Robert Muszyński},
   pdfsubject={Praca dyplomowa inżynierska - przykład i~wytyczne},
   pdfkeywords={bąk, Lagrange top, Euler top, praca dyplomowa, formatowanie, wytyczne}
}

%-------------------------| Geometria strony |-----------------------------

\geometry{
    top = 25mm,
    headheight = 15mm,
    headsep = 3mm,
    textheight = 24cm,
    textwidth = 16cm,
}

%---------------------| Frequently used commands definitions |-------------------

\newcommand{\red}{\color{red}}
\def\BibTeX{{\rm B\kern-.05em{\sc i\kern-.025em b}\kern-.08em
    T\kern-.1667em\lower.7ex\hbox{E}\kern-.125emX}}

\newtheorem{uwaga}{Uwaga}
\newtheorem{twr}{Twierdzenie}
%------------------------------------| END |-------------------------------------

%----------------------------| Math symbols definition |-------------------------

\newcommand{\angmom}{\boldsymbol{m}}
\newcommand{\bdvelo}{\boldsymbol{\omega}_B}
\newcommand{\grav}{\boldsymbol{g}}
\newcommand{\lagran}{L(\boldsymbol{q},\boldsymbol{\dot{q}})}
\newcommand{\COMvec}{\boldsymbol{r}_B}
\newcommand{\ee}{\boldsymbol{e}}
\newcommand{\FF}{\boldsymbol{F}}
\newcommand{\xx}{\boldsymbol{x}}
\newcommand{\qq}{\boldsymbol{q}}
\newcommand{\RR}{\boldsymbol{R}}
\newcommand{\TT}{\boldsymbol{T}}
\newcommand{\pp}{\boldsymbol{p}}
\newcommand{\iner}{\boldsymbol{I}_B}
\newcommand{\vv}{\boldsymbol{v}}
\newcommand{\bbs}{\boldsymbol}
\DeclareMathOperator{\const}{const}
\DeclareMathOperator*{\rank}{rank}  %star changes sub- and superscripts placement
%------------------------------------| END |-------------------------------------

%--------------------------| Ścieżki do rysunków |-------------------------------

\graphicspath{{figures/chapter_01/}{figures/chapter_02/}{figures/chapter_03/}{figures/chapter_04/}}

%--------| kontrola dołączanych plików - wygodne przy pracy nad tekstem |--------

%\includeonly{sources/Od_Autora,sources/02_czym_jest_bak}

%-------------------------| The document starts here |-------------------------------
\begin{document}

\pdfbookmark{Strona tytułowa}{tytul}
\maketitle
%% informacja o sposobie udostępniania tego dokumentu

%% dedykacja
\dedication{6cm}{Łukasz Stanisławowski\texttt{\textbackslash dedication}}

\cleardoublepage
\pdfbookmark{\contentsname}{Contents}
\tableofcontents            %spis treści
\markboth{\contentsname}{\contentsname}
%%\newpage
%\thispagestyle{empty}
%\cleardoublepage
%\thispagestyle{plain}

\mbox{}\vfill\hfill
\begin{minipage}{0.5\linewidth} 
  {\tiny \noindent Do składu pracy wykorzystano system przygotowania
    dokumentów~\LaTeX, opracowany przez
    L.~Lamporta\index{latex>\LaTeX} [Lam94], będący nakładką
    systemu \TeX, [Knu86a,Knu86b].  Matematyczne czcionki o nazwie
    {AMS Euler}, których używamy w tej pracy, zostały przygotowane
    przez H.\ Zapfa [KZ86], przy współpracy z~D.\ Knuthem i~jego
    studentami, na zlecenie Amerykańskiego Towarzystwa Matematycznego.
    %% Przy wybranej Antykwie Toruńskiej/Półtawskiego odznacz odpowiednio poniższe
    Wybrane czcionki składu tekstu, Antykwa Toruńska [Now97] -- jeden
    %Wybrane czcionki składu tekstu, Antykwa Półtawskiego [Now99] -- jeden
    z~nielicznych krojów pisma zaprojektowany specjalnie dla języka
    polskiego w~sposób uwzględniający jego rytm -- w~odczuciu autora
    doskonale współgrają z~kształtem czcionki {AMS Euler}, pozwalając
    na uzyskanie harmonijnej całości.
    % %% Przy wybranych czcionkach Concrete odznacz poniższe
    % Czcionki składu tekstu, zwane {Concrete Roman} i {Concrete
    %   Italic}, należące do knuthowskiej rodziny czcionek {Computer
    %   Modern}, zostały specjalnie przystosowane do kształtu czcionki
    % {AMS Euler} na potrzeby książki [GKP96].
    % %% Przy wybranych czcionkach URW Palladio odznacz poniższe
    % Czcionka składu tekstu, zwana URW Palladio jest klonem zapfoskiej rodziny
    % czcionek o~nazwie Palatino [LPn05] i~zdaniem autora świetnie współgra
    % z~kształtem czcionki {AMS Euler}.
    Składu bezszeryfowego tekstu maszynowego dokonano z~użyciem
    opracowanej przez R. Leviena czcionki o~nazwie Inconsolata
    [Lev15]\footnote{\red\tiny Chyba warto takie informacje szerzyć}.


\vspace{-4mm}

 \makeatletter
\renewenvironment{thebibliography}[1]
     {%
        \tiny%
      \list{\@biblabel{\@arabic\c@enumiv}}%
           {\settowidth\labelwidth{\@biblabel{#1}}%
\setlength{\itemsep}{2.5mm}
            \leftmargin\labelwidth
            \advance\leftmargin\labelsep
            \@openbib@code
            \usecounter{enumiv}%
            \let\p@enumiv\@empty
            \renewcommand\theenumiv{\@arabic\c@enumiv}}%
      \sloppy\clubpenalty4000\widowpenalty4000%
      \sfcode`\.\@m\vspace{5mm}}
     {\def\@noitemerr
       {\@latex@warning{Empty `thebibliography' environment}}%
      \endlist}

\makeatother

\begin{thebibliography}{Knu86b}

% %% Odmarkować pozycję gdy wybrane czcionki Concrete
% \bibitem[GKP96]{GKP96loc}
% R.~L. Graham, D.~E. Knuth i O.~Patashnik,
% \newblock { Matematyka konkretna}.
% \newblock PWN, Warszawa, 1996.\vspace{-3mm}

\bibitem[Knu86a]{Knuth86loc}
D.~E. Knuth,
\newblock { The \TeX book, volume {A} of Computers and Typesetting}.
\newblock Addison-Wesley, Reading, 1986.\vspace{-3mm}

\bibitem[Knu86b]{Knuth86aloc}
D.~E. Knuth,
\newblock { \TeX: {The} Program, volume {B} of Computers and Typesetting}.
\newblock Addison-Wesley, Reading, 1986.\vspace{-3mm}

\bibitem[KZ86]{KnZa89loc}
D.~E. Knuth i H.~Zapf,
\newblock {AMS} {Euler} --- {A} new typeface for mathematics.
\newblock { Scholary Publishing}, {20}:131--157, 1986.\vspace{-3mm}

\bibitem[Lam94]{Lamport94loc}
L.~Lamport,
\newblock { \LaTeX: A Document Preparation System}.
\newblock Addison-\mbox{-Wesley}, Reading, 1994.\vspace{-3mm}

\bibitem[Lev15]{Levien15loc}
R.~Levien,
\newblock {Inconsolata}.
\newblock \url{https://levien.com/type/myfonts/inconsolata.html}, 2015.\vspace{-3mm}

% %% Odmarkować pozycję przy wybranej czcionce URW Palladio
% \bibitem[LPn05]{LinotypePalatino05loc}
% Linotype Palatino nova: A classical typeface redesigned by Hermann Zapf,
% \newblock Linotype Library GmbH, 2005.\vspace{-2mm}

%% Odmarkować pozycję przy wybranej Antykwie Toruńskiej
\bibitem[Now97]{nowacki97loc}
J.~Nowacki,
\newblock {Antykwa} {Toruńska} -– od początku do końca polska czcionka.
\newblock {\em Biuletyn Polskiej Grupy Użytkowników Systemu \TeX}, 9:26--27,
  \nolinebreak1997.\vspace{-2mm}

% %% Odmarkować pozycję przy wybranej Antykwie Półtawskiego
% \bibitem[Now99]{nowacki99}
% J.~Nowacki,
% \newblock Piórkiem i {MetaPost-em}, czyli {Antykwa} {Półtawskiego}.
% \newblock {\em Biuletyn Polskiej Grupy Użytkowników Systemu \TeX}, 12:49--53,
%   \nolinebreak1999.\vspace{-2mm}

\end{thebibliography}
}
\end{minipage}
       %notka o systemie składu i czcionkach
% Side note about LaTeX and the fonts being used in the thesis
\mbox{}\vfill\hfill
\begin{minipage}{0.618033988749895\textwidth}
    \begin{refsection}[font-note-literature.bib] % separate bibliography for this minipage
                                                 % this resource was added with \addsectionbib
        {% smaller font for paragraph and bibliography
            \scriptsize%
            \renewcommand*{\bibfont}{\scriptsize}%
            \noindent%
            %
            For typesetting this thesis,
            the~\LaTeX{} document preparation system has been used.
            \LaTeX{} has been developed by L.~Lamport~\cite{lamport:latex},
            and~is an~overlay on top of the~\TeX{} system~\cite{knuth:texbook}.
            %
            Mathematical fonts called AMS~Euler which have been used in~this document,
            have been commissioned by the American Mathematical Society
            and~designed by H.\ Zapf~\cite{zapf:ams-euler} with the assistance of D.\ Knuth and his students.
            %
            The~URW~Palladio font, used for roman text,
            is a~clone of H.\ Zapf's old-style typeface called Palatino~\cite{zapf:palatino}.
            %
            Typesetting of sans-serif monospaced text has been done using
            Inconsolata font, created by R.\ Levien~\cite{levien:inconsolata}.
            %
            \noindent
            \printbibliography[heading=none,locallabelwidth=true] % print section bibliography
        }
    \end{refsection}
\end{minipage}
   %przykładowa wersja angielska notki

%% część właściwa dokumentu
\chapter{Wstęp}
\markboth{Wstęp}{Wstęp}
\label{ch_01}

dsfgsdfgsdfg
\chapter{Cel Pracy}
\markboth{Cel Pracy}{Cel Pracy}
\label{ch_02}
dfsbgdvbfdgb
% \include{src/02_czym_jest_bak}
% \include{sr/05_zakonczenie}

\cleardoublepage
\phantomsection
\addcontentsline{toc}{chapter}{\bibname}
\bibliographystyle{alphapl}
\bibliography{sr/bibliografia}
\markboth{\bibname}{\bibname}

%%spis tabel
%\cleardoublepage
%\phantomsection
%\listoftables
%\addcontentsline{toc}{chapter}{\listtablename}
%\markboth{\listtablename}{\listtablename}

%%spis rysunków
\cleardoublepage
\phantomsection
\addcontentsline{toc}{chapter}{\listfigurename}
\listoffigures
\markboth{\listfigurename}{\listfigurename}

\appendix
\include{sources/07_dodatek_bakiem_malowane}

%% lista rzeczy do zrobienia
\listoftodos

\end{document}
